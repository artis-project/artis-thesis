\chapter{Fundamentals}
\label{chap:technologies}
This chapter aims to introduce valuable background knowledge of the fundamentals built upon in this thesis.

\section{Internet of Things}
\gls{iot} technology does not have a single unique definition. However, \textcite[p. 165]{iot} defines the term Internet of Things as follows:
\begin{quote}
    "An open and comprehensive network of intelligent objects that have the capacity to auto-organize, share information, data, and resources, reacting and acting in face of situations and changes in the environment"
\end{quote}
While the internet in the traditional sense is about the data created by people, \gls{iot} is about data created by things. A practical example of this would be a smart heating system for a vacation home. Such a system is able to record and store data about environmental parameters such as temperature or humidity and make this data accessible in real-time on a smartphone through the internet. It is also possible to interact with this system through your phone. For example, raising the temperature of a winter vacation home the night before arrival would be possible. This thesis aims to utilize an \gls{iot} device capable of recording and storing environmental parameters to monitor an artwork in transit.

\section{Blockchain}
Blockchain technology has long been an evolution and promising advancement in distributed and decentralized systems. It describes a ledger that can be either distributed (permissioned) or decentralized (permissionless), tamper-evident, tamper-resistant and usually without a central authority. This technology allows a community of users to record transactions that cannot be changed once published \parencite{blockchainoverview}.

These properties can reduce the importance of trust among single parties, as the consensus of the whole network is necessary for a transaction to be published. This is why blockchain has been able to digitize processes that previously required trust in a central authority. The most prominent are cryptocurrencies like Bitcoin and Ethereum \cite{bitcoin, ethereum}.

\subsection{Smart Contracts}
Some blockchains can be extended and leveraged by smart contracts, essentially collections of code and data deployed on the blockchain network using cryptographically signed transactions. Nodes within the blockchain network execute the smart contract, and the results of execution are recorded on the blockchain. Users can create transactions that send data to public functions offered by a smart contract. The smart contract then executes the appropriate method to perform a service. Since the code is on the blockchain, it is tamper-evident and resistant, making it a trusted third party. Smart contracts can perform various functions such as calculations, storing information, exposing properties, and automatically sending funds to other accounts \parencite{blockchainoverview}.

\subsubsection{Non-Fungible Token}
An example of a smart contract standard would be the ERC-721 for \glspl{nft} on the Ethereum network~\cite{erc721}. \glspl{nft} are digital assets stored on a blockchain and represent a unique item or asset, such as a piece of artwork. 

\section{Tracking and Tracing}
Tracking and tracing of logistic networks is considered an important issue \cite{trackingtracing}. In this context, tracking refers to collecting and managing information about the current location and status of a product or delivery item \cite{trackingtracing}. Tracing on the other hand looks back in time and refers to the storing and retaining of a product or item's history \cite{practicetrackingtracing}.
