\chapter{Summary and Conclusions}
\label{chap:summary}
This chapter aims to conclude the thesis with a summary, conclusions, and potential future work based on the insights gained during development and the limitations of the system.

%Regarding Final Considerations:
% I. Some teachers and/or methods use the term Conclusion for a text at the end of the paper that aims to expose the results achieved, this term is not incorrect, but many of the works are bibliographical reviews where in the end no conclusion is obtained and yes Several considerations that were found in the development of written work. Therefore, in each project should be considered/weighted if there will be a Conclusion or Final Considerations. Usually what else happens is that you have Final Considerations. The Final Considerations of a paper aims to show if the goal sought for the project was achieved, as well as give a view of the most important considerations and conclusions on the subject addressed, among other aspects. This should include:
%1. An explanation stating clearly whether or not it has achieved the stated objectives (a subdivision between general objectives and specific objectives can also be made here). In each case the reasons must be explained:
%A. If you have achieved the objectives: inform the main factors that contributed to the success, describing them briefly, but do not leave doubts;
%B. If you have not met the objectives: inform how much of the objective has been achieved and cite the factors that contributed to the failure, describing them briefly, but that leaves no doubt.
%2. Describe the main considerations and conclusions that were obtained as a result of the execution of the work. Here should not be repeated text already in the work, but write the impressions of these considerations and how they contributed to the implementation and achieved the goal;
%3. Name and describe the main difficulties encountered in the execution of the work and project. All the work developed means an evolution for the student, and to reach this evolution, it has had to overcome a series of obstacles. Reporting obstacles and overcoming (or not overcoming) helps to dignify and show the merit of the work itself to the reader/evaluator. It is also a contribution, in the sense that once problems and solutions are exposed, readers/evaluators learn/know ways of solving or approaching such problems;
%4. Discuss whether modifications occurred during the execution of the work within the scope defined in the Project phase and in what was developed. It should be explained what generated those modifications, substantiating and justifying such changes.
%5. The relationship between the proposed schedule and the work schedule can be described. This allows the reader/evaluator to learn from the indicated distortions/hits.
%6. Describe or cite future work that may be done based on this work. During the execution of work, it is sought to reach a defined objective in the project. However, several interesting subjects of research are revealed (being that the same ones are not treated/researched in the work because they do not match the objective and scope of the work). The description of such subjects/themes/research demonstrates the students' perception of development as well as their vision of objectivity in the execution of this work.

%II. Conclusion / Final Considerations aim to show the reader/evaluator the student's perception of the work done. In this way, it is not advisable to do citations and references because theoretically everything that was necessary to quote and refer should already be done within the content of the work. Only in some very specific cases/situations can you make referrals or citations in this part of the work (this should be discussed thoroughly with the supervisor/teacher)

%%%%%%%%%%%%%%%%%%%
\section{Summary}
This thesis proposes a novel system to monitor and track artwork during transportation. The system is designed to include both blockchain and \gls{iot} to improve transparency and traceability in the art world. The idea is to attach an \gls{iot} device to the physical artwork that records and stores environmental data critical to the integrity of the artwork. The logging device evaluates the recorded data against a predefined threshold and alerts the stakeholders if any deviations occur. Any data associated with the artwork is stored on the blockchain to serve as an indisputable record of events. To create a proof of concept for this proposed system, a prototype was realized within the scope of this thesis.

The first step was to establish the theoretical background for this project through a literature review. The thesis introduced the concepts of blockchain, \gls{iot}, \glspl{sc}, and continued with a review of related work. The focus was placed on gathering existing solutions or work that use either blockchain, \gls{iot}, or both in tracking and monitoring systems. The results of the literature review were aggregated into a summary and used as inspiration for the design of the proposed system.

The prototype includes a user-friendly web application that serves as an interface to the developed \gls{api}. This \gls{rest} \gls{api} offers an abstracted interaction method with the \gls{sc} deployed on the blockchain. The system also features a monitoring device that is capable of logging temperature and humidity data and reporting any deviations from predefined thresholds to the server. The developed system presents a novel approach to artwork tracking and monitoring by integrating blockchain and \gls{iot}.

The design of the system includes a description of a simplified artwork tracking scenario that should be supported by the final prototype. The architecture of the system was then derived from this simplified scenario. The final system comprises a frontend user interface, a backend server, a \gls{sc}, and the \gls{iot} device. The frontend is designed to provide a user-friendly interface to the backend server. The backend server exposes a \gls{rest} \gls{api} to interact with the \gls{sc}. The developed \gls{sc} defines the unique digital counterpart of physical artwork as a \gls{nft}. By storing additional details on the \gls{sc}, it becomes feasible to register stakeholders as key actors. Their designated blockchain identities can then be used to access information about the artwork status or request modifications. Authorization is handled by the \gls{sc} itself through a role-based policy while authentication is handled on the server by challenging users to provide proof of ownership. The \gls{iot} device is designed to record temperature and humidity data and report violations of a predefined threshold via the \gls{api} to the \gls{sc}.

The established design and architecture were then implemented in the form of a prototype. The \gls{sc} was implemented in Solidity and deployed to the sepolia testnet. The backend server is written in Python and uses the web3py library to execute contract functions by initiating transactions. Those functions are then exposed as \gls{api} endpoints using the Flask framework. The backend server is deployed to the web as a Google Cloud-run service. The frontend is built using react and provides a \gls{gui} for the \gls{api} endpoints. Using Metamask the user is able to easily authenticate to the backend server by signing a challenge message and providing proof of ownership over the Ethereum account used. Lastly, to monitor humidity and temperature a rockPi was used with a DHT-22 sensor attached. The software for the logger is written in Python. 

The system was evaluated in terms of cost, performance, and security as well as a simulated artwork tracking scenario. The cost analysis was performed on the sepolia testnet and used the recorded gas consumption of function executions to estimate the transaction fees on the mainnet. The analysis also includes an estimation of the fees if the system was deployed on the Polygon network. To evaluate the performance of the system, the requests to the \gls{api} were timed multiple times and the resulting average was discussed. To evaluate the security of the system, a number of potential threat scenarios were analyzed and rated in terms of security risk.  The field test involved deploying a \gls{sc} instance, minting an \gls{nft}, registering the actors, requesting the artwork to be delivered, approving a status change, reporting a simulated violation, and finally delivering the artwork and changing the status to delivered. The transportation scenario included the logging device being transported outside for roughly an hour. 

\section{Conclusions}
A prototype system to monitor and track artwork using \gls{iot} and blockchain technology was successfully designed, implemented, and evaluated. The general objective has been clearly achieved by this prototype. The system is capable of creating an \gls{nft} as a digital counterpart of artwork. The owner of the artwork is then able to register other stakeholders and effectively give them read/write permissions to information about the artwork and its current status. By attaching a logging device to the artwork, the system is able to monitor environmental conditions and report any deviations to the system. Those violations are then irrefutably stored on the blockchain. Furthermore, the transaction history on the blockchain can be used to trace ownership and custody of the artwork. The system also provides a user-friendly frontend interface as well as an abstracted \gls{rest} \gls{api} to interact with the system and the underlying \gls{sc}.

The development of the system presented major challenges. For instance, authenticating to the system involves signing a challenge message with a private key. This process can be cumbersome to achieve without proper wallet integration. To overcome this challenge we decided to integrate the popular wallet provider Metamask in our frontend. With this integration, the authentication process is as simple as clicking a button. Another challenge was finding a suitable \gls{iot} device. Initially, the idea was to use a more sophisticated device with integrated sensors. The development on such a board turned out to be more complex and we decided to switch to a simpler device that is able to run on a familiar operating system like Linux. This choice resulted in a decline in the sensor's quality. Initially, the scope of this thesis also included protecting sensitive data by designing a confidentiality scheme using \gls{zkp}. We decided to leave this for future research as this was intended to be achieved by using existing libraries and frameworks that turned out to be harder to integrate than expected. Instead, we focused on a different set of features for the final prototype.

The developed prototype has been used in a simulated artwork transportation scenario and has been shown to work as intended. Nevertheless, the prototype has some limitations that have to be addressed in future work.

\section{Future Work}
This thesis has demonstrated a minimal prototype to demonstrate a novel approach to artwork tracking and monitoring in transportation scenarios. To arrive at a mature state of the system there is still great room for improvement and further development.

First, it would be essential to upgrade certain components of the system. This may include upgrading to a better sensor as well as adding more sensors to track location, vibration, and other environmental data. With an upgraded \gls{iot} device another field test performed at a larger scale will bring further insight into the feasibility of the system in a real-world scenario.

Secondly, future work should address the security risks outlined in Section \ref{sec:security_analysis}. Primarily the protection of sensitive data should be investigated. This may involve exploring the potential of using \glspl{zkp} to protect data exposed in blockchain transactions \cite{zkdet}.

Lastly, the current design includes a centralized server to provide an interface to the \gls{sc}. This introduces a single point of failure to the system and it might be worth exploring a truly distributed approach. 