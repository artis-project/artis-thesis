\chapter{Related Work}
In this chapter, we provide an overview of the existing literature related to art tracking with IoT and blockchains, highlighting previous research, methods, and findings related to the research question. Lastly, we also provide an overview of the state of the art in this field.

\section{Blockchain Technology}
Blockchain technology has been proposed as a promising solution to the challenges faced in the creative industry, such as issues related to the monetization of intellectual property but also the provenance and authenticity of creative work. \cite{creativeindustry} Especially technologies like smart contracts and \glspl{nft} have shown the potential of blockchain technology to revolutionize the art market by enabling greater transparency, accountability, and traceability. 

\section{NFTs in the Art World}
 \glspl{nft} have gained significant attention in the art world in recent years, with several high-profile sales of \gls{nft}-based artworks. \textcite{nftopportunities} has shown the benefits of \glspl{nft} protecting digital assets \textcite{creativeindustry} has suggested the application of \glspl{nft} beyond digital art, proposing to physically tag an \gls{iot} device to an artwork or sculpture. This could be used to transfer and track ownership while also potentially reducing intermediaries. \textcite{artchain} for example, developed a blockchain-based trading system for artworks. And \textcite{nftminter} has proposed a \gls{nft} Minter for a blockchain-based artwork trading system. Another related study by \textcite{zkdet} proposes a traceable and privacy-preserving data exchange scheme based on \gls{nft} and \gls{zkp}.

\section{Existing Art Tracking Systems Using IoT and Blockchain}
The combination of these two technologies in industrial systems and supply chains has been a "hot trend" in recent years. \cite{industryiot} Several companies and research groups are exploring the use of IoT and blockchains for art tracking and management.

For example, Artory \cite{artory} is tokenizing physical artworks to create a secure record of art pieces' provenance and ownership. \textcite{artrentalblockchain} developed an artwork rental system based on blockchain technology that is intended to facilitate the process of lending artwork collections. Another company that is incorporating \gls{iot} and blockchain technology is Everledger \cite{everledger}. Although they have been focused on other assets than artwork they also have discovered use cases that involve artwork tracking and authentication. Authena \cite{authena} is a swiss based company that is trying to protect the authenticity and traceability of physical assets. A study by \textcite{pactart} developed an \gls{iot} architecture called PACT-ART that employs advanced computing techniques like data mining and business process intelligence to predict a future state of the process and point out any possible violation from it.



