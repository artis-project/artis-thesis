%===============================================================================
%%%%%%%%%%%%%%%%%%%%%%%%%%%%%%%%
% Recipe to do an abstract:
%%%%%%%%%%%%%%%%%%%%%%%%%%%%%%%%
%(1) Context
%(2) Research Gap
%(3) Goals, Objectives
%(4) Methodology, how
%(5) Results

% Example:
%Abstract
%(1) In the context of energy efficiency in computer networks, a significant number of solutions ranging from protocols and functionalities to energy efficiency-oriented management applications have been proposed. 
%(2) However, the characteristics of environments to develop and validate such solutions are not as discussed as the solutions themselves. 
%(3) Considering this, this work proposes an emulation environment to develop and validate energy efficiency-oriented solutions, as well as discuss their specific characteristics. 
%(4) Thus, three functionalities of different network scopes are implemented, Adaptive Link Rate (interface level), Syncronized Coalescing (device level) and SustNMS (network level) in the Mininet emulation environment using the implemented software-defined networks paradigm on the POX controller. 
%(5) The environment is validated by comparing the energy savings achieved by these features in a topology inspired by the National Research Network (RNP).

\chapter*{Kurzfassung}
\addcontentsline{toc}{chapter}{Kurzfassung}
\selectlanguage{german}
Diese Arbeit befasst sich mit der Kombination von Internet of Things (IoT) und Blockchain-Technologien und konzentriert sich dabei auf die innovative Anwendung dieser Technologien im Bereich des Transports von Kunstwerken. Das zentrale Ziel ist die Einführung eines Systems, welches IoT und Blockchain nutzt, um die Überwachung und Verwaltung von Kunstwerken während des Transports zu verbessern.

Um dieses Ziel zu erreichen, wendet die Studie eine zweigleisige Methodik an. Zunächst wird eine umfassende umfassende Literaturrecherche durchgeführt, um ein grundlegendes Verständnis für die zugrundeliegenden Prinzipien. Anschließend wird ein angewandter Forschungsansatz ausgeführt, der die Entwicklung, Implementierung und Evaluierung eines Prototyps beinhaltet. Das Ergebnis dieser Forschungsarbeit ist ein funktionaler Prototyp, der
den angestrebten Anwendungsfall unterstützt. Es jedoch, sind weitere Schritte erforderlich zur Verfeinerung des Prototyps, insbesondere im Hinblick auf den Schutz sensibler Daten und die Optimierung der Genauigkeit.

Der Wert dieser Arbeit liegt in der innovativen Verschmelzung von IoT- und Blockchain-Technologien, die einen neuen Weg zur Bewältigung von Herausforderungen im Bereich der Kunstwerkstransportation demonstriert. Dies legt die Grundlage für zukünftige Bemühungen, dieses Konzept zu einer produktionsreifen Lösung weiterentwickeln.


\chapter*{Abstract}
\addcontentsline{toc}{chapter}{Abstract}
\selectlanguage{english}
This thesis delves into the convergence of the Internet of Things (IoT) and blockchain technologies, focusing on the innovative application of these technologies within the realm of artwork transportation. The central objective is to introduce a groundbreaking system that capitalizes on IoT and blockchain to enhance the tracking and management of artwork during transportation processes.

In pursuit of this goal, the study adopts a dual-pronged methodology. Initially, a comprehensive literature review is undertaken to establish a foundational understanding of the underlying principles. Subsequently, an applied research approach is employed, culminating in the design, implementation, and evaluation of a prototype tailored to the intricacies of artwork transportation.

The outcome of this research effort materializes as a functional prototype that effectively supports the targeted use case. However, it is acknowledged that further strides are needed to refine the prototype, particularly in safeguarding sensitive data and optimizing sensor accuracy.

The significance of this work lies in its innovative amalgamation of IoT and blockchain technologies, presenting a novel avenue for addressing challenges in the artwork transportation domain. By demonstrating the feasibility of such a system, this thesis lays the groundwork for future endeavors that seek to advance this concept into a production-ready solution.
