\chapter*{Abstract}

%===============================================================================
%%%%%%%%%%%%%%%%%%%%%%%%%%%%%%%%
% Recipe to do an abstract:
%%%%%%%%%%%%%%%%%%%%%%%%%%%%%%%%
%(1) Context
%(2) Research Gap
%(3) Goals, Objectives
%(4) Methodology, how
%(5) Results

% Example:
%Abstract
%(1) In the context of energy efficiency in computer networks, a significant number of solutions ranging from protocols and functionalities to energy efficiency-oriented management applications have been proposed. 
%(2) However, the characteristics of environments to develop and validate such solutions are not as discussed as the solutions themselves. 
%(3) Considering this, this work proposes an emulation environment to develop and validate energy efficiency-oriented solutions, as well as discuss their specific characteristics. 
%(4) Thus, three functionalities of different network scopes are implemented, Adaptive Link Rate (interface level), Syncronized Coalescing (device level) and SustNMS (network level) in the Mininet emulation environment using the implemented software-defined networks paradigm on the POX controller. 
%(5) The environment is validated by comparing the energy savings achieved by these features in a topology inspired by the National Research Network (RNP).


\addcontentsline{toc}{chapter}{Abstract}

\selectlanguage{german}

Das ist die Kurzfassung...



\selectlanguage{english}
