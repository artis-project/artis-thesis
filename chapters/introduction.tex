\chapter{Introduction}
The \gls{iot} has revolutionized the way we interact with technology and data, allowing us to seamlessly connect devices, sensors, and networks. At the same time, blockchain technology has emerged as a powerful tool for securely managing data and transactions in a decentralized, tamper-proof way. \gls{iot} devices such as sensors, generate massive amounts of data in many different use cases. By using blockchain technology to securely manage and store this data, authenticity, and integrity can be ensured. This combination of the two technologies can introduce new approaches that can facilitate or enhance current processes.

\section{CERTIFY Project}
\label{sec:certify}
Certify is a multi-partner research project \cite{certify}, dedicated to achieving a high level of security by developing a novel framework to manage security throughout the lifecycle of \gls{iot} devices. The project is scheduled to run from 1st October 2022 for 36 months and involves 13 partners from eight countries around Europe. The \gls{csg} of the \gls{ifi} at \gls{uzh} is part of certify and focusing on designing and developing a blockchain-based sharing infrastructure of security information, an Over-The-Air Patch Infrastructure, and the pilot for tracking and monitoring of artworks.

This thesis is part of the contributions by \gls{csg} to the pilot for tracking and monitoring of artworks.

\section{Motivation}
Artwork transportation has proven to be a logistical challenge in many ways \cite{artintransit}. Complying with regulatory laws of both the departure and destination country, taking out insurance on the artworks, using packaging that lowers the risk of damage, and many more. Keeping a record of all relevant transactions involves a lot of administrative paperwork as well as trust among the many parties involved. This thesis aims to design a system that can be used to manage and monitor the transportation of artworks. Using blockchain technology to carry out all the transactions involved, the system aims to mitigate the reliance on trust. To prevent leaking sensitive information, the thesis explores the usage of zero-knowledge proofs.

The author is motivated to explore the ways new technologies can improve established processes. In the example of artwork transportation, digital documentation and monitoring environmental parameters could be used to improve the overall experience for all involved parties. 

\section{Description of Work}
The work involves creating a secure and transparent system for tracking and monitoring the movement of a unique item, represented by a \gls{nft} \cite{nftminter}, using a combination of blockchain technology and \gls{iot} devices. The goal is to develop a system for monitoring and logging some environmental parameters, such as motion, and location, during the transportation of artworks, including zero-knowledge proof. The system should be able to define alert thresholds for these parameters and notify the artwork sender, carrier, and recipient of any anomalies, without revealing any sensitive data. The \gls{nft} is registered in a \gls{sc} by the item owner or holder and contains relevant information and regulations, such as ObjectID \cite{objectid}, as required by the \gls{icom}. The IoT Board Administrator sets the board to its initial state, ready to receive data from the \gls{iot} devices. A profile, consisting of a private and public key is generated and stored on the secure element of the device, ensuring that the private key remains confidential and cannot be leaked. The public key is registered on the blockchain and is associated with the \gls{nft} by the owner. This profile is able to update the state of the \gls{nft} by issuing transactions signed with the private key to the smart contract using the Secp256k1 algorithm. This allows the \gls{nft} to be updated with events such as "pick up at origin" and "delivery at destination", which are logged with relevant timestamps. Additionally, photo evidence is created upon pickup and before delivery to ensure the integrity and authenticity of the item. Multi-Signature operations are used to certify changes in responsibilities between different actors, such as when the Sender transfers responsibility for the artwork to the Carrier. This helps to ensure the secure and transparent transfer of custody of the item. Finally, the transfer of custody changes the behavior of the sensor from standby mode to constant monitoring mode, allowing for real-time tracking and monitoring of the item as it moves from one location to another. The system provides a secure and transparent way to track and monitor the movement of unique items, while also ensuring their authenticity and ownership.

\section{Thesis Outline}

