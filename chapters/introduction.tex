\chapter{Introduction}
\label{chap:introduction}

%The. Introduction:
%I. The introduction of a paper aims to show the objectives, the work and give a view of the subject addressed. This should include:
%1. Contextualization of the work as a whole. Within this context can be placed the justifications or motivations for the execution of the work (must be consistent with the arguments defined in the project);
%2. An overview of the whole subject matter at work, without going into details or drawing conclusions. This is intended to give an overview of all the content addressed;
%3. Definition of the objectives to be achieved with the work (they must be by what was defined in the project). In some cases (completion work or have 40 pages of content) it is recommended to subdivide the objectives into two categories:
%The. General objective: it is normally defined objective in the project with a bit more complementary information;
%B. Specific objectives: There are several intermediate objectives that must be achieved so that when all of these are achieved, it is possible to meet the general objective.
%4. Explain how the work is organized. This is not to describe the content of the chapter as it is in the index/summary, but to say how and why the work was organized and created in the way it is. It seeks to show the reader the logical meaning of evolution and allow it to judge the best way to read the work;
%5. Explain the method of performing the work. To carry out the work, the student can use several methods of work such as research / bibliographic review, field research, experimentation (empirical or not), etc. It is important to inform about the method because it indicates the reasoning and type of work performed.

%II. Looking at the described items that should compose the introduction, it 's hard to imagine that it has less than one page;

%III. Many teachers consider that 50% of the grade of a written work is related to Introduction and Conclusion / Final Considerations.


The \gls{iot} has revolutionized how we interact with technology and data, allowing us to connect devices, sensors, and networks seamlessly. At the same time, blockchain technology has emerged as a powerful tool for securely managing data and transactions in a decentralized, tamper-proof way. \gls{iot} devices, such as sensors, generate massive amounts of data in many different use cases. By using blockchain technology to manage and store this data, authenticity and integrity securely can be ensured. Combining the two technologies can introduce new approaches to facilitate or enhance current processes. 
%You need to expand this introduction and also use citations to build up your arguments. A straightforward manner is to use the thesis description as a starting point.

\section{CERTIFY Project}
\label{sec:certify}
CERTIFY is a multi-partner research project \cite{certify}, dedicated to achieving a high level of security by developing a novel framework to manage security throughout the lifecycle of \gls{iot} devices. The project is scheduled to run from 1st October 2022 for 36 months and involves 13 partners from eight European countries. The \gls{csg} of the \gls{ifi} at \gls{uzh} is part of CERTIFY and focusing on designing and developing a blockchain-based sharing infrastructure of security information, an Over-The-Air Patch Infrastructure, and the pilot for tracking and monitoring of artworks.

This thesis is part of the contributions by \gls{csg} to the pilot for tracking and monitoring of artworks. %Try to avoid paragraphs of one sentence. Typically, it either belongs to the previous paragraph or the central idea in the paragraph is not fully developed.

\section{Motivation}
Artwork transportation has proven to be a logistical challenge in many ways \cite{artintransit}. Complying with regulatory laws of both the departure and destination country, taking out insurance on the artworks, using packaging that lowers the risk of damage, and many more. Keeping a record of all relevant transactions typically involves administrative paperwork and trust among the many parties involved. %citation?

This thesis proposes a system that can be used to manage and monitor the transportation of artworks. Using blockchain technology to carry out all the transactions involved, the system aims to mitigate the reliance on trust. To prevent leaking sensitive information, the thesis explores the usage of zero-knowledge proofs.

This thesis further explores how new technologies can improve established processes. In the example of artwork transportation, digital documentation and monitoring environmental parameters could be used to improve the overall experience for all involved parties. 

\section{Description of Work}
The work involves creating a secure and transparent system for tracking and monitoring the movement of a unique item, represented by a \gls{nft} \cite{nftminter}, using a combination of blockchain technology and \gls{iot} devices. The goal is to develop a system for monitoring and logging some environmental parameters, such as temperature and humidity, while transporting artwork. The system should be able to define alert thresholds for these parameters and notify the artwork sender, carrier, and recipient of any anomalies. The \gls{nft} is registered in a \gls{sc} by the item owner or holder and contains relevant information and regulations, such as ObjectID \cite{objectid}, as required by the \gls{icom}.

The \gls{iot} Board Administrator sets the board to its initial state, ready to receive data from the sensors. A private and public key profile is generated and stored on the device, ensuring that the private key remains confidential and cannot be leaked. The public key is registered on the blockchain and is associated with the \gls{nft} by the owner. This profile can update the state of the \gls{nft} by issuing transactions signed with the private key to the smart contract. This allows the \gls{nft} to be updated with events such as "pick up at origin" and "delivery at destination", which are logged with relevant timestamps. 
Multi-approval operations certify changes in responsibilities between different actors, such as when the Sender transfers responsibility for the artwork to the Carrier. This helps to ensure the secure and transparent transfer of custody of the item. 

Finally, with the transfer of custody, the behavior of the sensor is set from standby mode to constant monitoring mode, allowing for real-time tracking and monitoring of the item as it moves from one location to another. The system provides a secure and transparent way to track and monitor unique items' movement while ensuring their authenticity and ownership.

\section{Thesis Outline}
Chapter \ref{chap:introduction} has provided introductory and motivational information. Chapter \ref{chap:technologies} gives a high-level overview of the theoretical background of the thesis. In Chapter \ref{chap:related_work} we discuss existing papers on the topic and elaborate on the current state-of-the-art regarding artwork tracking. This information is built upon in Chapter \ref{chap:architecture_design} when we present the architecture and design of our own solution in detail. The implementational aspects of the solution are presented in Chapter \ref{chap:implementation}. The implemented system is than evaluated in Chapter \ref{chap:evaluation} and final conclusions are drawn in Chapter \ref{chap:summary}.
