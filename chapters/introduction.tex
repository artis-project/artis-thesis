\chapter{Introduction}
\label{chap:introduction}
The art industry has remarkable diversity, encompassing a multitude of participants, including artists, collectors, auction houses, and art dealers. Alongside these stakeholders, many intermediaries, such as promoters, conservators, archivists, and curators, contribute to the industry. Nevertheless, at its core, the art world revolves around artistic creations, spanning from sculptures, installations, and canvases to more unconventional pieces.

In the realm of art, museums curate not only from their own collections but also from private art holdings or other institutions. Consequently, there arises a pressing need for reliable logistics partners capable of safeguarding art during transit, maintaining optimal conditions, and preventing any potential damage. This presents a promising opportunity for the utilization of IoT sensors, capable of monitoring factors like temperature, humidity, vibrations, and other environmental factors  \cite{certifydeliverable}. Furthermore, the integration of blockchain technology holds the potential to guarantee the accuracy and security of the transportation process for all involved parties. It also extends its utility into the realm of art trading, enabling the verification and documentation of virtual and physical ownership transfers. In doing so, it introduces an element of standardization into a market that has traditionally been characterized by its lack of regulation and opacity \cite{certifydeliverable}.


\section{CERTIFY Project}
\label{sec:certify}
CERTIFY is a multi-partner research project \cite{certify}, dedicated to achieving a high level of security by developing a novel framework to manage security throughout the lifecycle of \gls{iot} devices. The project is scheduled to run from 1st October 2022 for 36 months and involves 13 partners from eight European countries. The \gls{csg} of the \gls{ifi} at \gls{uzh} is part of CERTIFY and focusing on designing and developing a blockchain-based sharing infrastructure of security information, an Over-The-Air Patch Infrastructure and a pilot project for tracking and monitoring artworks during transportation. This thesis is part of the contributions by \gls{csg} to this pilot project.

\section{Motivation}
Artwork transportation has proven to be a logistical challenge in many ways \cite{artintransit}. Complying with regulatory laws of both the departure and destination country, taking out insurance on the artworks, using packaging that lowers the risk of damage, and many more. Recording all relevant transactions typically involves administrative paperwork and trust among the many parties involved \cite{artintransit}.

The \gls{iot} has revolutionized how we interact with technology and data, allowing us to connect devices, sensors, and networks seamlessly. At the same time, blockchain technology has emerged as a powerful tool for securely managing data and transactions in a decentralized, tamper-proof way. Combined, these technologies offer exciting possibilities for creating new forms of digital art that are both secure and transparent.

\gls{iot} devices, such as sensors, can record data that can be used to monitor the environment around an artwork. By using blockchain technology to manage and store this data securely, stakeholders can ensure the integrity of artwork even in a trustless environment. The blockchain ledger can additionally improve documentation of ownership and custody by indefinitely storing transactions on the chain.

In this context, the thesis proposes a system for artwork tracking in combination with hardware and software deployed to provide automatic monitoring and management of artwork in a transportation scenario.

\section{Description of Work}
The work involves creating a secure and transparent system for tracking and monitoring the movement of a unique item, represented by a \gls{nft} \cite{nftminter}, using a combination of blockchain technology and \gls{iot} devices. The goal is to develop a system for monitoring and logging some environmental parameters, such as temperature and humidity, while transporting artwork. The system should be able to define alert thresholds for these parameters and notify the artwork sender, carrier, and recipient of any anomalies. The \gls{nft} is registered in a \gls{sc} by the item owner or holder and contains relevant information and regulations, such as ObjectID \cite{objectid}, as required by the \gls{icom}.

The \gls{iot} Board Administrator sets the board to its initial state, ready to receive data from the sensors. A private and public key profile is generated by creating a blockchain account and stored on the device, ensuring the private key remains confidential and cannot be leaked. The account address is associated with the \gls{nft} by the owner. This account can update the state of the \gls{nft} by issuing transactions signed with the private key to the smart contract. This allows the \gls{nft} to be updated with events such as "pick up at origin" and "delivery at destination," which are logged with relevant timestamps. Multi-approval operations certify changes in responsibilities between different actors, such as when the Sender transfers responsibility for the artwork to the Carrier. This helps to ensure the secure and transparent transfer of custody of the item. 

Finally, with the transfer of custody, the behavior of the sensor is set from standby mode to constant monitoring mode, allowing for real-time tracking and monitoring of the item as it moves from one location to another. The system provides a secure and transparent way to track and monitor unique items' movement while ensuring their authenticity and ownership.

\subsection{Thesis Goals}
\label{sec:thesis_goals}
\textbf{Research:} The research report should include a review of the relevant literature concerning the technologies used and include existing solutions for blockchain-based artwork tracking.

\textbf{Design on Solution Architecture:} The report should include detailed specifications and requirements for the system, including the backend and frontend.

\textbf{Solution Prototyping:} This goal covers the implementation of a prototype of the artwork tracking system. This involves working with \gls{iot} devices, sensors, and other suitable technologies. The prototype should be functional and demonstrate the key features of the system.

\textbf{Evaluation:} The system evaluation involves testing the prototype in a (simulated) real-world artwork transportation scenario. The evaluation report includes an analysis of the system's performance and an assessment of its real-world usability.

\section{Methodology}
The methodology adopted in this thesis is strategically designed to comprehensively address the objectives outlined in Section \ref{sec:thesis_goals}. This approach can be categorized into two primary phases: a literature review phase and an applied research phase.

\textbf{Literature review phase:} This involves conducting a literature review to gather essential knowledge about fundamental concepts and related work in artwork tracking and blockchain and \gls{iot}. This review provides an overview of related work on tracking solutions, summarized in Section \ref{sec:related_work_discussion}. The knowledge gained serves as a basis for the design and development of the proposed system.
\clearpage
\textbf{Applied research phase:} The applied research phase is composed of creating and evaluating a prototype for the proposed system. It includes the design, implementation, and evaluation described below.

\begin{itemize}[font=\itshape, align=left, itemindent=0.5cm]
    \item[Design:] The system architecture design is based on a simplified artwork tracking scenario. The prototype is required to support this scenario concerning the goals defined in Section \ref{sec:thesis_goals}. The architecture includes all necessary components and interactions to support the use case. This phase delivers a documentation of this design presented in Chapter \ref{chap:architecture_design}.
    \item[Implementation:] The output of the previous phase needs to be implemented as a prototype system. The implementation process and details regarding the implementation of features are reported in Chapter \ref{chap:implementation}. 
    \item[Evaluation:] In the final phase, the delivered prototype of the implementation phase is evaluated. The evaluation includes a cost and performance analysis outlined in Section \ref{sec:cost_and_performance}, a security analysis in Section \ref{sec:security_analysis}, and a test run in a simulated artwork tracking scenario in Section \ref{sec:field_test}. The results are discussed in Section \ref{sec:eval_discussion}.
\end{itemize}


\section{Thesis Outline}
Chapter \ref{chap:introduction} has provided introductory and motivational information. Chapter \ref{chap:technologies} gives a high-level overview of the theoretical background of the thesis. In Chapter \ref{chap:related_work}, we discuss existing papers on the topic and elaborate on the current systems regarding artwork tracking. This information is built upon in Chapter \ref{chap:architecture_design} when we present the architecture and design of our solution in detail. The implementational aspects of the solution are presented in Chapter \ref{chap:implementation}. The implemented system is then evaluated in Chapter \ref{chap:evaluation}, and a summary and conclusions are given in Chapter \ref{chap:summary}. Furthermore, Chapter \ref{chap:summary} addresses the potential for future work and suggests opportunities for improvement of the proposed system.
